\section{Parallel matrix multiplication}\label{sec:multiplication}

The first task asks for the implementation of a parallel matrix by matrix multiplication %
algorithm in \textit{C} using OpenMP. The goal of the task is to check the scalability %
of the code and evaluate its efficiency.

Matrix multiplication methods vary in performance when applied to dense and sparse %
matrices \cite{algorithms}. The most trivial way to calculate this operation is %
the "row by column" method which involves computing each element of the resulting %
matrix by taking the dot product of a row of the first matrix and the corresponding %
column of the second one. It is commonly used for dense matrices and requires $n^3$ %
multiplications for a $n\times n$ matrix. While it also works for sparse matrices, %
it is less efficient due to the large portions of zero elements.\\%
Sparse matrices offer advantages in memory usage and computation efficiency by %
storing only the non-zero values and their positions, reducing the occupied space %
\cite{ds-algorithms}. Moreover, sparse matrix-oriented algorithms, like Strassen's, %
exploit the structure of the data to reduce the number of operations needed while %
also benefitting from a better cache utilization.\\%
In my assessment, implementations of different algorithms for the two types of %
matrices and the respective analyses are beyond the scope of this assignment. I %
choose to implement the "row by column" algorithm that works for both, even if %
it may be less performant.

% The most trivial method is called "row by column" and involves computing each element of the %
% resulting matrix by taking the dot product of a row of the first matrix and the %
% corresponding column of the second one. Other algorithms and %
% techniques are more complex\cite{algorithms, ds-algorithms} and exploit the hardware %
% of the machine (registers, cache and memory alignment, etc.) or the content of %
% the matrices to get better performance. Such methodologies are %
% "row by row", "column by column", "Strassen's", and many others.\\%

% For the sake of simplicity, I choose to implement the "row by column" method. %
% It was designed with dense matrices in mind but works fine with sparse matrices %
% without exploiting their benefits, like reduced memory usage and data reutilization. %
% I opted for this method as, in my assessment, the more complex ones surpass the %
% scope of this assignment.

The algorithm was implemented sequentially and later I introduced %
OpenMP directives to obtain a parallel code.\\%
Source code \ref{code:multiplication} shows the parallelized function with the %
directives used. The choice of which ones to use was made after reading %
the professor's material on the subject \cite{prof-slides}, OpenMP's 3.1 %
documentation \cite{openmp-cs, openmp-docs}, and trying various combinations. %
In the end, I decided to use the following: %
\texttt{\#pragma omp parallel for collapse(2) reduction(+:temp)}. In particular:
\begin{itemize}
    \item \texttt{\#pragma omp parallel}: creates a parallel region in which all %
        code is distributed among the specified number of CPUs. The amount is %
        controlled externally by the \texttt{OMP\_NUM\_THREADS} environment %
        variable.
    \item \texttt{for}: \texttt{for}-loops are going to be distributed between available %
        CPUs.
    \item \texttt{collapse(2)} applies to the previous \texttt{for} directive and %
        specifies the number of nested loops to divide among the CPUs.
    \item \texttt{reduction(+:temp)} specifies the variable and the operation on %
        which to perform the reduction.
\end{itemize}

\begin{tabularx}{\textwidth}{@{} c c c c || c c c c @{}}
\caption{\label{table:multiplication}Matrix multiplication - Run times and FLOPS}\\
\toprule
    \textbf{Size} & \textbf{Cores} & \textbf{Run Time [s]} & \textbf{GFLOPS} & %
    \textbf{Size} & \textbf{Cores} & \textbf{Run Time [s]} & \textbf{GFLOPS}\\
\midrule
\endhead
    \multirow{6}{*}{200} & 1  & 0.0272 & 0.59 & \multirow{6}{*}{1200} & 1  & 7.9264 & 0.44  \\
                         & 2  & 0.0150 & 1.06 &                       & 2  & 4.1837 & 0.83  \\
                         & 4  & 0.0076 & 2.10 &                       & 4  & 2.1500 & 1.61  \\
                         & 8  & 0.0041 & 3.86 &                       & 8  & 1.0419 & 3.32  \\
                         & 16 & 0.0025 & 6.39 &                       & 16 & 0.5124 & 6.74  \\
                         & 32 & 0.0026 & 6.24 &                       & 32 & 0.2546 & 13.57 \\
\midrule
    \multirow{6}{*}{400} & 1  & 0.2509 & 0.51  & \multirow{6}{*}{1400} & 1  & 14.1042 & 0.39  \\
                         & 2  & 0.2405 & 0.53  &                       & 2  & 6.9411  & 0.79  \\
                         & 4  & 0.0656 & 1.95  &                       & 4  & 3.9240  & 1.40  \\
                         & 8  & 0.0424 & 3.02  &                       & 8  & 1.9400  & 2.83  \\
                         & 16 & 0.0178 & 7.19  &                       & 16 & 0.9362  & 5.86  \\
                         & 32 & 0.0097 & 13.19 &                       & 32 & 0.4717  & 11.64 \\
\midrule
    \multirow{6}{*}{600} & 1  & 0.9862 & 0.44  & \multirow{6}{*}{1600} & 1  & 26.1967 & 0.31  \\
                         & 2  & 0.5122 & 0.84  &                       & 2  & 11.8077 & 0.69  \\
                         & 4  & 0.2474 & 1.75  &                       & 4  & 6.4396  & 1.27  \\
                         & 8  & 0.1273 & 3.39  &                       & 8  & 3.1493  & 2.60  \\
                         & 16 & 0.0629 & 6.87  &                       & 16 & 1.5955  & 5.13  \\
                         & 32 & 0.0331 & 13.03 &                       & 32 & 0.8052  & 10.17 \\
\midrule
    \multirow{6}{*}{800} & 1  & 2.3327 & 0.44  & \multirow{6}{*}{1800} & 1  & 37.6769 & 0.31 \\
                         & 2  & 1.1591 & 0.88  &                       & 2  & 18.6574 & 0.63 \\
                         & 4  & 0.5950 & 1.72  &                       & 4  & 9.5689  & 1.22 \\
                         & 8  & 3.3135 & 3.27  &                       & 8  & 5.1028  & 2.29 \\
                         & 16 & 0.1514 & 6.76  &                       & 16 & 2.4839  & 4.70 \\
                         & 32 & 0.0769 & 12.87 &                       & 32 & 1.3263  & 8.79 \\
\midrule
    \multirow{6}{*}{1000} & 1  & 4.6862 & 0.43  & \multirow{6}{*}{2000} & 1  & 48.0570 & 0.33 \\
                          & 2  & 2.2981 & 0.87  &                       & 2  & 25.5150 & 0.63 \\
                          & 4  & 1.1806 & 1.69  &                       & 4  & 12.4870 & 1.28 \\
                          & 8  & 0.5962 & 3.35  &                       & 8  & 6.7303  & 2.38 \\
                          & 16 & 0.2936 & 6.81  &                       & 16 & 3.4271  & 4.67 \\
                          & 32 & 0.1483 & 13.49 &                       & 32 & 1.7600  & 9.09 \\
\bottomrule
\end{tabularx}


Before the final run on the University's HPC cluster (GCC 4.8.5, OpenMP 3.1), %
the code was tested and debugged on my local machine, a laptop equipped with an %
Intel\textsuperscript{\textregistered} Core\textsuperscript{\texttrademark} %
i5-8300H at 2.30GHz and 16GB of RAM running Fedora 39 (GCC 13.2.1, OpenMP 4.5).\\%
As an effort to simplify the compilation and execution of the various iterations %
of the code with different values, I wrote a \textit{bash} script which later %
became the PBS job submission file. This script is responsible for creating the %
necessary directories on the file system, for compilation, and execution with %
the various values. The results are returned in the form of a series of %
\texttt{.CSV} files which are easier to work on with other programs.

\subsection*{Results analysis}
The implemented algorithm was run with dense square matrices of sizes ranging %
from 200 to 2000 with up to 32 CPUs. The node responsible for the computation was %
\texttt{hpc-c11-node22.unitn.it} equipped with an Intel\textsuperscript{\textregistered} %
Xeon\textsuperscript{\textregistered} Gold 6252N CPU running at 2.30GHz and 256GB %
of RAM. The obtained execution times are reported in Table \ref{table:multiplication} %
alongside the FLOPS calculated using Equation \ref{eq:flops} and expressed as %
GFLOPS.

\begin{equation}
    \label{eq:flops}
    \textnormal{FLOPS}=\frac{%
            2 \cdot \textnormal{A rows} \cdot \textnormal{A columns} \cdot %
            \textnormal{B columns}
        }{\textnormal{execution time}}=
        \frac{%
            2 \cdot \textnormal{size}^3}{\textnormal{execution time}}
\end{equation}

\begin{tabularx}{\textwidth}{@{} c c c c || c c c c @{}}
\caption{\label{table:multiplication}Matrix multiplication - Run times and FLOPS}\\
\toprule
    \textbf{Size} & \textbf{Cores} & \textbf{Run Time [s]} & \textbf{GFLOPS} & %
    \textbf{Size} & \textbf{Cores} & \textbf{Run Time [s]} & \textbf{GFLOPS}\\
\midrule
\endhead
    \multirow{6}{*}{200} & 1  & 0.0272 & 0.59 & \multirow{6}{*}{1200} & 1  & 7.9264 & 0.44  \\
                         & 2  & 0.0150 & 1.06 &                       & 2  & 4.1837 & 0.83  \\
                         & 4  & 0.0076 & 2.10 &                       & 4  & 2.1500 & 1.61  \\
                         & 8  & 0.0041 & 3.86 &                       & 8  & 1.0419 & 3.32  \\
                         & 16 & 0.0025 & 6.39 &                       & 16 & 0.5124 & 6.74  \\
                         & 32 & 0.0026 & 6.24 &                       & 32 & 0.2546 & 13.57 \\
\midrule
    \multirow{6}{*}{400} & 1  & 0.2509 & 0.51  & \multirow{6}{*}{1400} & 1  & 14.1042 & 0.39  \\
                         & 2  & 0.2405 & 0.53  &                       & 2  & 6.9411  & 0.79  \\
                         & 4  & 0.0656 & 1.95  &                       & 4  & 3.9240  & 1.40  \\
                         & 8  & 0.0424 & 3.02  &                       & 8  & 1.9400  & 2.83  \\
                         & 16 & 0.0178 & 7.19  &                       & 16 & 0.9362  & 5.86  \\
                         & 32 & 0.0097 & 13.19 &                       & 32 & 0.4717  & 11.64 \\
\midrule
    \multirow{6}{*}{600} & 1  & 0.9862 & 0.44  & \multirow{6}{*}{1600} & 1  & 26.1967 & 0.31  \\
                         & 2  & 0.5122 & 0.84  &                       & 2  & 11.8077 & 0.69  \\
                         & 4  & 0.2474 & 1.75  &                       & 4  & 6.4396  & 1.27  \\
                         & 8  & 0.1273 & 3.39  &                       & 8  & 3.1493  & 2.60  \\
                         & 16 & 0.0629 & 6.87  &                       & 16 & 1.5955  & 5.13  \\
                         & 32 & 0.0331 & 13.03 &                       & 32 & 0.8052  & 10.17 \\
\midrule
    \multirow{6}{*}{800} & 1  & 2.3327 & 0.44  & \multirow{6}{*}{1800} & 1  & 37.6769 & 0.31 \\
                         & 2  & 1.1591 & 0.88  &                       & 2  & 18.6574 & 0.63 \\
                         & 4  & 0.5950 & 1.72  &                       & 4  & 9.5689  & 1.22 \\
                         & 8  & 3.3135 & 3.27  &                       & 8  & 5.1028  & 2.29 \\
                         & 16 & 0.1514 & 6.76  &                       & 16 & 2.4839  & 4.70 \\
                         & 32 & 0.0769 & 12.87 &                       & 32 & 1.3263  & 8.79 \\
\midrule
    \multirow{6}{*}{1000} & 1  & 4.6862 & 0.43  & \multirow{6}{*}{2000} & 1  & 48.0570 & 0.33 \\
                          & 2  & 2.2981 & 0.87  &                       & 2  & 25.5150 & 0.63 \\
                          & 4  & 1.1806 & 1.69  &                       & 4  & 12.4870 & 1.28 \\
                          & 8  & 0.5962 & 3.35  &                       & 8  & 6.7303  & 2.38 \\
                          & 16 & 0.2936 & 6.81  &                       & 16 & 3.4271  & 4.67 \\
                          & 32 & 0.1483 & 13.49 &                       & 32 & 1.7600  & 9.09 \\
\bottomrule
\end{tabularx}


The parallel scalability of the algorithm was evaluated using the FLOPS as the %
performance metric. Plot \ref{plot:mult-performance}, Plot \ref{plot:mult-speedup}, %
and Plot \ref{plot:mult-efficiency} allow us to see a visual representation of the %
performance, the speedup, and the efficiency, respectively.

\begin{figure}[h!tb]
    \centering
    \captionsetup{type=plot}
    \caption{\label{plot:mult-performance}Parallel Matrix Multiplication - Performance graph by size of matrices}
    \begin{tikzpicture}
        \begin{axis}[
            title={},
            xlabel={CPUs},
            ylabel={GFLOPS},
            legend pos=outer north east,
            xtick={1,2,4,8,16,32},
            log ticks with fixed point,
            xmode=log,
            cycle multiindex* list={
                color list
                    \nextlist
                marks
                    \nextlist
            },
        ]
            \addplot
                coordinates {
                    (1, 0.59) 
                    (2, 1.06) 
                    (4, 2.10) 
                    (8, 3.86) 
                    (16, 6.39)
                    (32, 6.24)
                };
            
            \addplot
                coordinates {
                    (1, 0.51) 
                    (2, 0.53) 
                    (4, 1.95) 
                    (8, 3.02) 
                    (16, 7.19) 
                    (32, 13.19)
                };
            
            \addplot
                coordinates {
                    (1, 0.44) 
                    (2, 0.84) 
                    (4, 1.75) 
                    (8, 3.39) 
                    (16, 6.87)
                    (32, 13.03)
                };
            
            \addplot
                coordinates {
                    (1, 0.44) 
                    (2, 0.88) 
                    (4, 1.72) 
                    (8, 3.27) 
                    (16, 6.76) 
                    (32, 12.87)
                };
            
            \addplot
                coordinates {
                    (1, 0.43) 
                    (2, 0.87) 
                    (4, 1.69) 
                    (8, 3.35) 
                    (16, 6.81) 
                    (32, 13.49)
                };

            \addplot
                coordinates {
                    (1, 0.44) 
                    (2, 0.83) 
                    (4, 1.61) 
                    (8, 3.32) 
                    (16, 6.74) 
                    (32, 13.57)
                };
            
            \addplot
                coordinates {
                    (1, 0.39) 
                    (2, 0.79) 
                    (4, 1.40) 
                    (8, 2.83) 
                    (16, 5.86) 
                    (32, 11.64)
                };

            \addplot
                coordinates {
                    (1, 0.31) 
                    (2, 0.69) 
                    (4, 1.27) 
                    (8, 2.60) 
                    (16, 5.13) 
                    (32, 10.17)
                };
            
            \addplot
                coordinates {
                    (1, 0.31) 
                    (2, 0.63) 
                    (4, 1.22) 
                    (8, 2.29) 
                    (16, 4.70) 
                    (32, 8.79)
                };
            
            \addplot
                coordinates {
                    (1, 0.33) 
                    (2, 0.63) 
                    (4, 1.28) 
                    (8, 2.38) 
                    (16, 4.67) 
                    (32, 9.09)
                };

            \legend{200, 400, 600, 800, 1000, 1200, 1400, 1600, 1800, 2000}
        \end{axis}
    \end{tikzpicture}
\end{figure}
\begin{figure}[h!tb]
    \centering
    \captionsetup{type=plot}
    \caption{\label{plot:mult-speedup}Parallel Matrix Multiplication - Speedup graph by size of matrices}
    \begin{tikzpicture}
        \begin{axis}[
            title={},
            xlabel={CPUs},
            ylabel={Speedup $[\times]$},
            legend pos=outer north east,
            xtick={1,2,4,8,16,32},
            log ticks with fixed point,
            xmode=log,
            %ymode=log,
            cycle multiindex* list={
                color list
                    \nextlist
                marks
                    \nextlist
            },
        ]
            \addplot
                coordinates {
                    (1,1)
                    (2,1.81)
                    (4,3.58)
                    (8,6.57)
                    (16,10.88)
                    (32,10.62)
                };
            
            \addplot
                coordinates {
                    (1,1)
                    (2,1.04)
                    (4,3.82)
                    (8,5.92)
                    (16,14.09)
                    (32,25.84)
                };
            
            \addplot
                coordinates {
                    (1,1)
                    (2,1.93)
                    (4,3.99)
                    (8,7.75)
                    (16,15.68)
                    (32,29.75)
                };
            
            \addplot
                coordinates {
                    (1,1)
                    (2,2.01)
                    (4,3.92)
                    (8,7.44)
                    (16,15.41)
                    (32,29.32)
                };
            
            \addplot
                coordinates {
                    (1,1)
                    (2,2.04)
                    (4,3.97)
                    (8,7.86)
                    (16,15.96)
                    (32,31.60)
                };

            \addplot
                coordinates {
                    (1,1)
                    (2,1.89)
                    (4,3.69)
                    (8,7.61)
                    (16,15.47)
                    (32,31.13)
                };
            
            \addplot
                coordinates {
                    (1,1)
                    (2,2.03)
                    (4,3.59)
                    (8,7.27)
                    (16,15.06)
                    (32,29.90)
                };
            
            \addplot
                coordinates {
                    (1,1)
                    (2,2.22)
                    (4,4.07)
                    (8,8.32)
                    (16,16.42)
                    (32,32.54)
                };
            
            \addplot
                coordinates {
                    (1,1)
                    (2,2.02)
                    (4,3.94)
                    (8,7.38)
                    (16,15.17)
                    (32,28.41)
                };
            
            \addplot
                coordinates {
                    (1,1)
                    (2,1.88)
                    (4,3.85)
                    (8,7.14)
                    (16,14.02)
                    (32,27.30)
                };

            \legend{200, 400, 600, 800, 1000, 1200, 1400, 1600, 1800, 2000}
        \end{axis}
    \end{tikzpicture}
\end{figure}
\begin{figure}[h!tb]
    \centering
    \captionsetup{type=plot}
    \caption{\label{plot:mult-efficiency}Parallel Matrix Multiplication - Efficiency graph by size of matrices}
    \begin{tikzpicture}
        \begin{axis}[
            title={},
            xlabel={CPUs},
            ylabel={Efficiency $[\%]$},
            legend pos=outer north east,
            xtick={1,2,4,8,16,32},
            xmode=log,
            ymode=log,
            log ticks with fixed point,
            cycle multiindex* list={
                color list
                    \nextlist
                marks
                    \nextlist
            },
        ]
            \addplot
                coordinates {
                    (1,100)
                    (2,90.67)
                    (4,89.54)
                    (8,82.11)
                    (16,67.99)
                    (32,33.20)
                };
            
            \addplot
                coordinates {
                    (1,100)
                    (2,52.17)
                    (4,95.55)
                    (8,74.05)
                    (16,88.05)
                    (32,80.76)
                };
            
            \addplot
                coordinates {
                    (1,100)
                    (2,96.28)
                    (4,99.64)
                    (8,96.86)
                    (16,98.02)
                    (32,92.98)
                };
            
            \addplot
                coordinates {
                    (1,100)
                    (2,100.62)
                    (4,98.01)
                    (8,93)
                    (16,96.31)
                    (32,91.61)
                };
            
            \addplot
                coordinates {
                    (1,100)
                    (2,101.96)
                    (4,99.24)
                    (8,98.25)
                    (16,99.75)
                    (32,98.74)
                };

            \addplot
                coordinates {
                    (1,100)
                    (2,94.73)
                    (4,92.17)
                    (8,95.10)
                    (16,96.68)
                    (32,97.29)
                };
            
            \addplot
                coordinates {
                    (1,100)
                    (2,101.60)
                    (4,89.86)
                    (8,90.88)
                    (16,94.16)
                    (32,93.45)
                };
            
            \addplot
                coordinates {
                    (1,100)
                    (2,110.93)
                    (4,101.70)
                    (8,103.98)
                    (16,102.62)
                    (32,101.67)
                };
            
            \addplot
                coordinates {
                    (1,100)
                    (2,100.97)
                    (4,98.44)
                    (8,92.29)
                    (16,94.80)
                    (32,88.77)
                };
            
            \addplot
                coordinates {
                    (1,100)
                    (2,94.17)
                    (4,96.21)
                    (8,89.25)
                    (16,87.64)
                    (32,85.33)
                };

            \legend{200, 400, 600, 800, 1000, 1200, 1400, 1600, 1800, 2000}
        \end{axis}
    \end{tikzpicture}
\end{figure}

It is easy to see that the execution has a better performance (Plot \ref{plot:mult-performance}) %
the more CPUs it can utilize because of the higher number of operations it can %
execute per second. The same consideration is also valid for the speedup %
(Plot \ref{plot:mult-speedup}): the more CPUs, the faster the calculations are %
executed. \\
The spikes observable in the efficiency graph (plot \ref{plot:mult-efficiency}) %
are because parallel systems are nonlinearly scalable. We must choose %
either a constant run time or a constant efficiency when scaling up the problem %
\cite{scalability}. The general downward slope can be interrupted by undefined %
peaks caused by imbalance problems. In case in exam, it is particularly evident with %
$200\times200$ matrices (red line) and $400\times400$ matrices (blue line).
