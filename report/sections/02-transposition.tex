\section{Parallel matrix transposition}\label{sec:transposition}

The second task asks for the implementation of a matrix transposition algorithm in %
two different ways, written in \textit{C} and parallelized with OpenMP. The end %
goal is to compare the scalability of the two versions and evaluate their efficiency.

Transposing a matrix involves flipping it over its main diagonal, exchanging %
rows and columns, resulting in a new matrix where the rows of the original one %
become the columns of the transposed matrix and vice versa. The first algorithm %
does verbatim what described above, while the other divides the original %
matrix into blocks of fixed sizes, transposes them using the same principle, %
and finally transposes each block's content.

Both algorithms were written and tested sequentially and parallelized later %
by introducing the OpenMP directives visible in source code \ref{code:transposition}. %
The choice of which ones to use was made after reading the professor's material %
on the subject \cite{prof-slides}, OpenMP's 3.1 documentation \cite{openmp-cs, openmp-docs}, %
and trying various combinations\footnote{For an explanation of each part of the directive, %
refer to \hyperref[sec:multiplication]{Task \ref*{sec:multiplication}, Parallel %
matrix multiplication}.}.

\begin{tabularx}{\textwidth}{@{} c c c c c c c c @{}}
    \caption{\label{table:transposition}Matrix transposition - run times and bandwidth}\\
    \toprule
        \textbf{Size} & \textbf{Cores} & \textbf{Run 1 [s]} & \textbf{Run 2 [s]} & %
        \textbf{Run 3 [s]} & \textbf{Run 4 [s]} & \textbf{Average [s]} & \textbf{Bandwidth [GB/s]}\\
    \midrule
    \endhead
        \multirow{6}{*}{20000} & 1  & 6.5606 & 6.6204 & 6.6265 & 6.5600 & 6.5919 & 0.24 \\
                               & 2  & 3.4106 & 3.3133 & 3.3296 & 3.3078 & 3.3404 & 0.48 \\
                               & 4  & 2.0776 & 1.8602 & 2.0055 & 2.0716 & 2.0037 & 0.80 \\
                               & 8  & 0.9174 & 0.9229 & 0.9655 & 0.9833 & 0.9473 & 1.69 \\
                               & 16 & 0.5285 & 0.5373 & 0.5404 & 0.5687 & 0.5437 & 2.94 \\
                               & 32 & 0.3506 & 0.3423 & 0.3365 & 0.3830 & 0.3532 & 4.53 \\
    \midrule
        \multirow{6}{*}{30000} & 1  & 15.6335 & 15.5900 & 15.5903 & 16.2771 & 15.7727 & 0.23 \\
                               & 2  & 8.4734  & 8.1278  & 8.2218  & 8.2349  & 8.2645  & 0.44 \\
                               & 4  & 4.3465  & 4.8519  & 4.9496  & 4.8396  & 4.7469  & 0.76 \\
                               & 8  & 2.2628  & 2.4801  & 2.6046  & 2.2887  & 2.4091  & 1.49 \\
                               & 16 & 1.2341  & 1.8231  & 1.2558  & 1.2322  & 1.3863  & 2.60 \\
                               & 32 & 0.8240  & 0.8335  & 1.1785  & 0.7841  & 0.9050  & 3.98 \\
    \midrule
        \multirow{6}{*}{40000} & 1  & 27.9708 & 28.9593 & 27.9213 & 27.9582 & 28.2024 & 0.23 \\
                               & 2  & 14.9170 & 14.8816 & 14.8110 & 14.9570 & 14.8919 & 0.43 \\
                               & 4  & 8.4412  & 8.3528  & 8.6755  & 8.3326  & 8.4505  & 0.76 \\
                               & 8  & 4.3593  & 4.4382  & 4.2855  & 4.1860  & 4.3172  & 1.48 \\
                               & 16 & 3.2731  & 3.0261  & 2.2626  & 2.5232  & 2.7712  & 2.31 \\
                               & 32 & 3.1133  & 2.7508  & 2.5513  & 1.9700  & 2.5964  & 2.46 \\
    \midrule
        \multirow{6}{*}{50000} & 1  & 45.9321 & 44.8595 & 45.1216 & 47.1102 & 45.7558 & 0.22 \\
                               & 2  & 23.8948 & 25.4520 & 25.4178 & 25.2524 & 25.0042 & 0.40 \\
                               & 4  & 13.8669 & 13.7983 & 13.6129 & 13.5577 & 13.7090 & 0.73 \\
                               & 8  & 8.6068  & 7.5582  & 8.0404  & 8.4306  & 8.1590  & 1.23 \\
                               & 16 & 4.1942  & 5.3581  & 5.3738  & 4.5863  & 4.8781  & 2.05 \\
                               & 32 & 3.3619  & 4.7703  & 3.2806  & 3.4283  & 3.7103  & 2.70 \\
    \bottomrule
\end{tabularx}

Compilation and execution are achieved, both locally (see previous task for the %
architecture) and on the HPC cluster, via the same PBS script used for task %
\ref*{sec:multiplication}, integrated with the new commands. The outputs of the %
runs are returned in \texttt{CSV} files for easier analysis with external programs.

\subsection*{Results analysis}
The implemented code was executed with square matrices of sizes between 20000 %
and 50000, each with square blocks of 1000, 2000, 5000, and 10000. All combinations %
were run with up to 32 CPUs on \texttt{hpc-c11-node22.unitn.it}\footnote{refer to %
\hyperref[sec:multiplication]{task \ref*{sec:multiplication}} for its architecture.} %
Execution times are reported in Table \ref{table:transposition} and Table %
\ref{table:transposition-blocks} alongside the achieved bandwidth, calculated %
using Equation \ref{eq:bandwidth}. Due to the way I wrote the \textit{C} code, %
the algorithm without the blocks runs four times per matrix size per number of %
CPUs. The bandwidth reported in Table \ref{table:transposition} refers to the %
average of the runtimes.\\%

\begin{equation}
    \label{eq:bandwidth}
    \textnormal{Bandwidth}=
        \frac{%
            2 \cdot \textnormal{Rows} \cdot \textnormal{Columns} \cdot %
            \textnormal{size\_of(\textit{float})}
            }
            {\textnormal{execution time}}=\frac{2\cdot \textnormal{size}^2\cdot 4}
                                               {\textnormal{execution time}}
        \qquad \textnormal{[B/s]}
\end{equation}

\begin{tabularx}{\textwidth}{@{} c c c c c c c c @{}}
    \caption{\label{table:transposition}Matrix transposition - run times and bandwidth}\\
    \toprule
        \textbf{Size} & \textbf{Cores} & \textbf{Run 1 [s]} & \textbf{Run 2 [s]} & %
        \textbf{Run 3 [s]} & \textbf{Run 4 [s]} & \textbf{Average [s]} & \textbf{Bandwidth [GB/s]}\\
    \midrule
    \endhead
        \multirow{6}{*}{20000} & 1  & 6.5606 & 6.6204 & 6.6265 & 6.5600 & 6.5919 & 0.24 \\
                               & 2  & 3.4106 & 3.3133 & 3.3296 & 3.3078 & 3.3404 & 0.48 \\
                               & 4  & 2.0776 & 1.8602 & 2.0055 & 2.0716 & 2.0037 & 0.80 \\
                               & 8  & 0.9174 & 0.9229 & 0.9655 & 0.9833 & 0.9473 & 1.69 \\
                               & 16 & 0.5285 & 0.5373 & 0.5404 & 0.5687 & 0.5437 & 2.94 \\
                               & 32 & 0.3506 & 0.3423 & 0.3365 & 0.3830 & 0.3532 & 4.53 \\
    \midrule
        \multirow{6}{*}{30000} & 1  & 15.6335 & 15.5900 & 15.5903 & 16.2771 & 15.7727 & 0.23 \\
                               & 2  & 8.4734  & 8.1278  & 8.2218  & 8.2349  & 8.2645  & 0.44 \\
                               & 4  & 4.3465  & 4.8519  & 4.9496  & 4.8396  & 4.7469  & 0.76 \\
                               & 8  & 2.2628  & 2.4801  & 2.6046  & 2.2887  & 2.4091  & 1.49 \\
                               & 16 & 1.2341  & 1.8231  & 1.2558  & 1.2322  & 1.3863  & 2.60 \\
                               & 32 & 0.8240  & 0.8335  & 1.1785  & 0.7841  & 0.9050  & 3.98 \\
    \midrule
        \multirow{6}{*}{40000} & 1  & 27.9708 & 28.9593 & 27.9213 & 27.9582 & 28.2024 & 0.23 \\
                               & 2  & 14.9170 & 14.8816 & 14.8110 & 14.9570 & 14.8919 & 0.43 \\
                               & 4  & 8.4412  & 8.3528  & 8.6755  & 8.3326  & 8.4505  & 0.76 \\
                               & 8  & 4.3593  & 4.4382  & 4.2855  & 4.1860  & 4.3172  & 1.48 \\
                               & 16 & 3.2731  & 3.0261  & 2.2626  & 2.5232  & 2.7712  & 2.31 \\
                               & 32 & 3.1133  & 2.7508  & 2.5513  & 1.9700  & 2.5964  & 2.46 \\
    \midrule
        \multirow{6}{*}{50000} & 1  & 45.9321 & 44.8595 & 45.1216 & 47.1102 & 45.7558 & 0.22 \\
                               & 2  & 23.8948 & 25.4520 & 25.4178 & 25.2524 & 25.0042 & 0.40 \\
                               & 4  & 13.8669 & 13.7983 & 13.6129 & 13.5577 & 13.7090 & 0.73 \\
                               & 8  & 8.6068  & 7.5582  & 8.0404  & 8.4306  & 8.1590  & 1.23 \\
                               & 16 & 4.1942  & 5.3581  & 5.3738  & 4.5863  & 4.8781  & 2.05 \\
                               & 32 & 3.3619  & 4.7703  & 3.2806  & 3.4283  & 3.7103  & 2.70 \\
    \bottomrule
\end{tabularx}
\begin{tabularx}{\textwidth}{@{} c c c c c @{}}
    \caption{\label{table:transposition-blocks}Matrix transposition in blocks - run times and bandwidth}\\
\toprule
    \textbf{Size} & \textbf{Block size} & \textbf{Cores} & \textbf{Run time [s]} & \textbf{Bandwidth [GB/s]}\\
\midrule
\endhead
    \multirow{24}{*}{20000} & \multirow{6}{*}{1000} & 1  & 2.3274 & 171.87  \\
                            &                       & 2  & 1.3910 & 287.56  \\
                            &                       & 4  & 1.0382 & 385.29  \\
                            &                       & 8  & 0.3582 & 1116.72 \\
                            &                       & 16 & 0.1652 & 2421.84 \\
                            &                       & 32 & 0.0894 & 4473.82 \\
                            \cmidrule{2-5}
                            & \multirow{6}{*}{2000} & 1  & 2.9142 & 137.26  \\
                            &                       & 2  & 1.5654 & 255.53  \\
                            &                       & 4  & 1.3991 & 285.90  \\
                            &                       & 8  & 0.4399 & 909.24  \\
                            &                       & 16 & 0.1971 & 2029.22 \\
                            &                       & 32 & 0.1029 & 3886.55 \\
                            \cmidrule{2-5}
                            & \multirow{6}{*}{5000} & 1  & 3.3928 & 117.90  \\
                            &                       & 2  & 1.9972 & 200.28  \\
                            &                       & 4  & 2.1841 & 183.14  \\
                            &                       & 8  & 0.5174 & 773.12  \\
                            &                       & 16 & 0.2396 & 1669.54 \\
                            &                       & 32 & 0.1340 & 2985.25 \\
                            \cmidrule{2-5}
                            & \multirow{6}{*}{10000} & 1  & 5.6197 & 71.18   \\
                            &                        & 2  & 2.8618 & 139.77  \\
                            &                        & 4  & 2.0601 & 194.16  \\
                            &                        & 8  & 0.8063 & 496.08  \\
                            &                        & 16 & 0.3628 & 1102.46 \\
                            &                        & 32 & 0.1888 & 2118.68 \\
\midrule\enlargethispage{-\baselineskip}
    \multirow{24}{*}{30000} & \multirow{6}{*}{1000} & 1  & 4.8689 & 184.85  \\
                            &                       & 2  & 2.9412 & 306.00  \\
                            &                       & 4  & 1.3300 & 676.71  \\
                            &                       & 8  & 1.0382 & 866.91  \\
                            &                       & 16 & 0.4314 & 2086.13 \\
                            &                       & 32 & 0.1845 & 4877.28 \\
                            \cmidrule{2-5}
                            & \multirow{6}{*}{2000} & 1  & 7.0540 & 127.59  \\
                            &                       & 2  & 3.6089 & 249.38  \\
                            &                       & 4  & 2.8164 & 319.56  \\
                            &                       & 8  & 1.6397 & 548.88  \\
                            &                       & 16 & 0.9963 & 903.34  \\
                            &                       & 32 & 0.2399 & 3750.95 \\
                            \cmidrule{2-5}
                            & \multirow{6}{*}{5000} & 1  & 7.8568 & 114.55  \\
                            &                       & 2  & 4.2641 & 211.06  \\
                            &                       & 4  & 3.2850 & 273.97  \\
                            &                       & 8  & 1.2891 & 698.14  \\
                            &                       & 16 & 0.5993 & 1501.79 \\
                            &                       & 32 & 0.3935 & 2287.10 \\
                            \cmidrule{2-5}
                            & \multirow{6}{*}{10000} & 1  & 13.1201 & 68.60   \\
                            &                        & 2  & 7.1527  & 125.83  \\
                            &                        & 4  & 5.2946  & 169.98  \\
                            &                        & 8  & 2.2368  & 402.35  \\
                            &                        & 16 & 0.9249  & 9733.03 \\
                            &                        & 32 & 0.4554  & 1976.34 \\
\midrule\enlargethispage{-\baselineskip}
    \multirow{24}{*}{40000} & \multirow{6}{*}{1000} & 1  & 8.7225 & 183.43  \\
                            &                       & 2  & 5.3019 & 301.78  \\
                            &                       & 4  & 4.8171 & 332.15  \\
                            &                       & 8  & 2.8220 & 566.97  \\
                            &                       & 16 & 0.9196 & 1739.83 \\
                            &                       & 32 & 0.5399 & 2963.67 \\
                            \cmidrule{2-5}
                            & \multirow{6}{*}{2000} & 1  & 12.2732 & 130.37  \\
                            &                       & 2  & 6.6707  & 239.85  \\
                            &                       & 4  & 4.0177  & 398.23  \\
                            &                       & 8  & 2.1598  & 740.81  \\
                            &                       & 16 & 1.1351  & 1409.55 \\
                            &                       & 32 & 0.6754  & 2368.82 \\
                            \cmidrule{2-5}
                            & \multirow{6}{*}{5000} & 1  & 13.4611 & 118.68  \\
                            &                       & 2  & 7.4703  & 214.18  \\
                            &                       & 4  & 6.8054  & 235.11  \\
                            &                       & 8  & 2.4402  & 655.69  \\
                            &                       & 16 & 1.0151  & 1576.22 \\
                            &                       & 32 & 0.8766  & 1825.21 \\
                            \cmidrule{2-5}
                            & \multirow{6}{*}{10000} & 1  & 24.0736 & 66.46   \\
                            &                        & 2  & 12.3900 & 129.14  \\
                            &                        & 4  & 9.0544  & 176.71  \\
                            &                        & 8  & 4.3846  & 364.91  \\
                            &                        & 16 & 1.7084  & 936.55  \\
                            &                        & 32 & 0.8562  & 1868.69 \\
\midrule\enlargethispage{-\baselineskip}
    \multirow{24}{*}{50000} & \multirow{6}{*}{1000} & 1  & 12.0920 & 206.75  \\
                            &                       & 2  & 7.6869  & 325.23  \\
                            &                       & 4  & 5.0051  & 499.49  \\
                            &                       & 8  & 3.0580  & 817.54  \\
                            &                       & 16 & 1.3538  & 1846.62 \\
                            &                       & 32 & 0.6621  & 3775.99 \\
                            \cmidrule{2-5}
                            & \multirow{6}{*}{2000} & 1  & 20.3203 & 123.03  \\
                            &                       & 2  & 10.6947 & 233.76  \\
                            &                       & 4  & 8.7424  & 285.96  \\
                            &                       & 8  & 4.4028  & 567.81  \\
                            &                       & 16 & 1.6551  & 1510.47 \\
                            &                       & 32 & 1.1047  & 2263.04 \\
                            \cmidrule{2-5}
                            & \multirow{6}{*}{5000} & 1  & 21.5054 & 116.25  \\
                            &                       & 2  & 11.5888 & 215.73  \\
                            &                       & 4  & 7.6854  & 325.29  \\
                            &                       & 8  & 4.1386  & 604.07  \\
                            &                       & 16 & 2.0332  & 1229.56 \\
                            &                       & 32 & 0.8638  & 2894.14 \\
                            \cmidrule{2-5}
                            & \multirow{6}{*}{10000} & 1  & 37.0605 & 67.46   \\
                            &                        & 2  & 19.4318 & 128.66  \\
                            &                        & 4  & 13.5048 & 185.12  \\
                            &                        & 8  & 6.8553  & 364.68  \\
                            &                        & 16 & 2.8058  & 891.01  \\
                            &                        & 32 & 1.2871  & 1942.34 \\
\bottomrule
\end{tabularx}


The parallel scalability of the algorithm was evaluated using the bandwidth as %
the performance metric. Plot \ref{plot:trans-performance}, Plot \ref{plot:trans-speedup}, %
and Plot \ref{plot:trans-efficiency} visualize the performance, the speedup, and %
the efficiency respectively. 

\begin{figure}[h!tb]%
    \centering
    \captionsetup{type=plot}
    \caption{\label{plot:trans-performance}Parallel Matrix Transposition - Performance graph by size of matrices}
    \begin{tikzpicture}%
        \begin{axis}[
            title={without blocks},
            xlabel={CPUs},
            ylabel={Bandwidth [GB/s]},
            legend pos=north west,
            xtick={1,2,4,8,16,32},
            log ticks with fixed point,
            xmode=log,
            ymode=log,
            cycle multiindex* list={
                color list
                    \nextlist
                marks
                    \nextlist
            },
            width={0.5\linewidth},
        ]%
            \addplot
                coordinates {
                    (1,60.68)
                    (2,119.75)
                    (4,199.63)
                    (8,422.26)
                    (16,735.70)
                    (32,1132.77)
                };
            
            \addplot
                coordinates {
                    (1,57.06)
                    (2,108.90)
                    (4,189.60)
                    (8,373.59)
                    (16,649.21)
                    (32,994.44)
                };
            
            \addplot
                coordinates {
                    (1,56.73)
                    (2,107.44)
                    (4,189.34)
                    (8,370.61)
                    (16,577.36)
                    (32,616.24)
                };
            
            \addplot
                coordinates {
                    (1,54.64)
                    (2,99.98)
                    (4,182.36)
                    (8,306.41)
                    (16,512.49)
                    (32,673.80)
                };

            \legend{20000, 30000, 40000, 50000}
        \end{axis}
    \end{tikzpicture}
    \begin{tikzpicture}%
        \begin{groupplot}[
            group style={
                group size=2 by 2,
                group name=plot,
                xlabels at=edge bottom,
                ylabels at=edge left,
                vertical sep=1.5cm,
                horizontal sep=1.5cm,
            }, 
            width={0.45\linewidth},
            xlabel={CPUs},
            ylabel={Bandwidth [GB/s]},
            xtick={1,2,4,8,16,32},
            legend to name=legend,
            legend columns=-1,
            cycle multiindex* list={
                color list
                    \nextlist
                marks
                    \nextlist
            },
        ]
            \nextgroupplot[
                title={$20000\times20000$},
                xmode=log, 
                ymode=log,
                log ticks with fixed point,
            ]

                \addplot
                    coordinates {
                        (1,171.87)
                        (2,287.56)
                        (4,385.29)
                        (8,1116.72)
                        (16,2421.84)
                        (32,4473.82)
                    };
        
                \addplot
                    coordinates {
                        (1,137.26)
                        (2,255.53)
                        (4,285.90)
                        (8,909.24)
                        (16,2029.22)
                        (32,3886.55)
                    };
                
                \addplot
                    coordinates {
                        (1,117.90)
                        (2,200.28)
                        (4,183.14)
                        (8,773.12)
                        (16,1669.54)
                        (32,2985.25)
                    };
                
                \addplot
                    coordinates {
                        (1,71.18)
                        (2,139.77)
                        (4,194.16)
                        (8,496.08)
                        (16,1102.46)
                        (32,2118.68)
                    };

            \nextgroupplot[
                title={$30000\times30000$},
                xmode=log, 
                ymode=log,
                log ticks with fixed point,
            ]
                \addplot
                    coordinates {
                        (1,184.85)
                        (2,306)
                        (4,676.71)
                        (8,866.91)
                        (16,2086.13)
                        (32,4877.28)
                    };
                
                \addplot
                    coordinates {
                        (1,127.59)
                        (2,249.38)
                        (4,319.56)
                        (8,548.88)
                        (16,903.34)
                        (32,3750.95)
                    };
                
                \addplot
                    coordinates {
                        (1,114.55)
                        (2,211.06)
                        (4,273.97)
                        (8,698.14)
                        (16,1501.79)
                        (32,2287.10)
                    };
                
                \addplot
                    coordinates {
                        (1,68.60)
                        (2,125.83)
                        (4,169.98)
                        (8,402.35)
                        (16,973.03)
                        (32,1976.34)
                    };    

            \nextgroupplot[
                title={$40000\times40000$},
                xmode=log, 
                ymode=log,
                log ticks with fixed point,
            ]
                \addplot
                    coordinates {
                        (1,183.43)
                        (2,301.78)
                        (4,332.15)
                        (8,566.97)
                        (16,1739.83)
                        (32,2963.67)
                    };
                
                \addplot
                    coordinates {
                        (1,130.37)
                        (2,239.85)
                        (4,398.23)
                        (8,740.81)
                        (16,1409.55)
                        (32,2368.82)
                    };
                
                \addplot
                    coordinates {
                        (1,118.86)
                        (2,214.18)
                        (4,235.11)
                        (8,655.69)
                        (16,1576.22)
                        (32,1825.21)
                    };
                
                \addplot
                    coordinates {
                        (1,66.46)
                        (2,129.14)
                        (4,176.71)
                        (8,364.91)
                        (16,936.55)
                        (32,1868.69)
                    };
            
            \nextgroupplot[
                title={$50000\times50000$},
                xmode=log, 
                ymode=log,
                log ticks with fixed point,
            ]
                \addplot
                    coordinates {
                        (1,206.75)
                        (2,325.23)
                        (4,499.49)
                        (8,817.54)
                        (16,1846.62)
                        (32,3775.99)
                    };
                
                \addplot
                    coordinates {
                        (1,123.03)
                        (2,233.76)
                        (4,285.96)
                        (8,567.81)
                        (16,1510.47)
                        (32,2263.04)
                    };
                
                \addplot
                    coordinates {
                        (1,116.25)
                        (2,215.73)
                        (4,325.29)
                        (8,604.07)
                        (16,1229.56)
                        (32,2894.14)
                    };
                
                \addplot
                    coordinates {
                        (1,67.46)
                        (2,128.66)
                        (4,185.12)
                        (8,364.68)
                        (16,891.01)
                        (32,1976.34)
                    };

            \legend{1000, 2000, 5000, 10000}
        \end{groupplot}
        \node (title) at ($(plot c1r1.center)!0.5!(plot c2r1.center)+(0,3cm)$) {with blocks};
    \end{tikzpicture}
    \ref{legend}
\end{figure}
\begin{figure}[h!tb]%
    \centering
    \captionsetup{type=plot}
    \caption{\label{plot:trans-speedup}Parallel Matrix Transposition - Speedup graph by size of matrices}
    \begin{tikzpicture}%
        \begin{semilogxaxis}[
            title={without blocks},
            xlabel={CPUs},
            ylabel={Speedup $[\times]$},
            legend pos=north west,
            xtick={1,2,4,8,16,32},
            log ticks with fixed point,
            cycle multiindex* list={
                color list
                    \nextlist
                marks
                    \nextlist
            },
            width={0.5\linewidth},
        ]%
            \addplot
                coordinates {
                    (1,1)
                    (2,1.97)
                    (4,3.29)
                    (8,6.96)
                    (16,12.12)
                    (32,18.67)
                };
            
            \addplot
                coordinates {
                    (1,1)
                    (2,1.91)
                    (4,3.32)
                    (8,6.55)
                    (16,11.38)
                    (32,17.43)
                };
            
            \addplot
                coordinates {
                    (1,1)
                    (2,1.89)
                    (4,3.34)
                    (8,6.53)
                    (16,10.18)
                    (32,10.86)
                };
            
            \addplot
                coordinates {
                    (1,1)
                    (2,1.83)
                    (4,3.34)
                    (8,5.61)
                    (16,9.38)
                    (32,12.33)
                };

            \legend{20000, 30000, 40000, 50000}
        \end{semilogxaxis}
    \end{tikzpicture}
    \begin{tikzpicture}%
        \begin{groupplot}[
            group style={
                group size=2 by 2,
                group name=plot,
                xlabels at=edge bottom,
                ylabels at=edge left,
                vertical sep=1.5cm,
                horizontal sep=1.5cm,
            }, 
            width={0.45\linewidth},
            xlabel={CPUs},
            ylabel={Speedup $[\times]$},
            xtick={1,2,4,8,16,32},
            legend to name=legend,
            legend columns=-1,
            cycle multiindex* list={
                color list
                    \nextlist
                marks
                    \nextlist
            },
        ]
            \nextgroupplot[
                title={$20000\times20000$},
                xmode=log, 
                log ticks with fixed point,
            ]

                \addplot
                    coordinates {
                        (1,1)
                        (2,1.67)
                        (4,2.24)
                        (8,6.50)
                        (16,14.09)
                        (32,26.03)
                    };
        
                \addplot
                    coordinates {
                        (1,1)
                        (2,1.86)
                        (4,2.08)
                        (8,6.62)
                        (16,14.78)
                        (32,28.32)
                    };
                
                \addplot
                    coordinates {
                        (1,1)
                        (2,1.70)
                        (4,1.55)
                        (8,6.56)
                        (16,14.16)
                        (32,25.32)
                    };
                
                \addplot
                    coordinates {
                        (1,1)
                        (2,1.96)
                        (4,2.73)
                        (8,6.97)
                        (16,15.49)
                        (32,29.77)
                    };

            \nextgroupplot[
                title={$30000\times30000$},
                xmode=log, 
                log ticks with fixed point,
            ]
                \addplot
                    coordinates {
                        (1,1)
                        (2,1.66)
                        (4,3.66)
                        (8,4.69)
                        (16,11.29)
                        (32,26.39)
                    };
                
                \addplot
                    coordinates {
                        (1,1)
                        (2,1.95)
                        (4,2.50)
                        (8,4.30)
                        (16,7.08)
                        (32,29.40)
                    };
                
                \addplot
                    coordinates {
                        (1,1)
                        (2,1.84)
                        (4,2.39)
                        (8,6.09)
                        (16,13.11)
                        (32,19.97)
                    };
                
                \addplot
                    coordinates {
                        (1,1)
                        (2,1.83)
                        (4,2.48)
                        (8,5.87)
                        (16,14.18)
                        (32,28.81)
                    };    

            \nextgroupplot[
                title={$40000\times40000$},
                xmode=log, 
                log ticks with fixed point,
            ]
                \addplot
                    coordinates {
                        (1,1)
                        (2,1.65)
                        (4,1.81)
                        (8,3.09)
                        (16,9.48)
                        (32,16.16)
                    };
                
                \addplot
                    coordinates {
                        (1,1)
                        (2,1.84)
                        (4,3.05)
                        (8,5.68)
                        (16,10.81)
                        (32,18.17)
                    };
                
                \addplot
                    coordinates {
                        (1,1)
                        (2,1.80)
                        (4,1.98)
                        (8,5.52)
                        (16,13.26)
                        (32,15.36)
                    };
                
                \addplot
                    coordinates {
                        (1,1)
                        (2,1.94)
                        (4,2.66)
                        (8,5.49)
                        (16,14.09)
                        (32,28.12)
                    };
            
            \nextgroupplot[
                title={$50000\times50000$},
                xmode=log, 
                log ticks with fixed point,
            ]
                \addplot
                    coordinates {
                        (1,1)
                        (2,1.57)
                        (4,2.42)
                        (8,3.95)
                        (16,8.93)
                        (32,18.26)
                    };
                
                \addplot
                    coordinates {
                        (1,1)
                        (2,1.90)
                        (4,2.32)
                        (8,4.62)
                        (16,12.28)
                        (32,18.39)
                    };
                
                \addplot
                    coordinates {
                        (1,1)
                        (2,1.86)
                        (4,2.80)
                        (8,5.20)
                        (16,10.58)
                        (32,24.90)
                    };
                
                \addplot
                    coordinates {
                        (1,1)
                        (2,1.91)
                        (4,2.74)
                        (8,5.41)
                        (16,13.21)
                        (32,28.79)
                    };

            \legend{1000, 2000, 5000, 10000}
        \end{groupplot}
        \node (title) at ($(plot c1r1.center)!0.5!(plot c2r1.center)+(0,3cm)$) {with blocks};
    \end{tikzpicture}
    \ref{legend}
\end{figure}
\begin{figure}[h!tb]%
    \centering
    \captionsetup{type=plot}
    \caption{\label{plot:trans-efficiency}Parallel Matrix Transposition - Efficiency graph by size of matrices}
    \begin{tikzpicture}%
        \begin{axis}[
            title={without blocks},
            xlabel={CPUs},
            ylabel={Efficiency [\%]},
            legend pos=south west,
            xtick={1,2,4,8,16,32},
            xmode=log,
            ymode=log,
            log ticks with fixed point,
            cycle multiindex* list={
                color list
                    \nextlist
                marks
                    \nextlist
            },
            width={0.5\linewidth},
        ]%
            \addplot
                coordinates {
                    (1,100)
                    (2,98.67)
                    (4,82.24)
                    (8,86.98)
                    (16,75.78)
                    (32,58.34)
                };
            
            \addplot
                coordinates {
                    (1,100)
                    (2,95.42)
                    (4,83.07)
                    (8,81.84)
                    (16,71.11)
                    (32,54.46)
                };
            
            \addplot
                coordinates {
                    (1,100)
                    (2,94.69)
                    (4,83.43)
                    (8,81.66)
                    (16,63.61)
                    (32,33.94)
                };
            
            \addplot
                coordinates {
                    (1,100)
                    (2,91.50)
                    (4,83.44)
                    (8,70.10)
                    (16,58.62)
                    (32,38.54)
                };

            \legend{20000, 30000, 40000, 50000}
        \end{axis}
    \end{tikzpicture}
    \begin{tikzpicture}%
        \begin{groupplot}[
            group style={
                group size=2 by 2,
                group name=plot,
                xlabels at=edge bottom,
                ylabels at=edge left,
                vertical sep=1.5cm,
                horizontal sep=1.5cm,
            }, 
            width={0.45\linewidth},
            xlabel={CPUs},
            ylabel={Efficiency [\%]},
            xtick={1,2,4,8,16,32},
            legend to name=legend,
            legend columns=-1,
            cycle multiindex* list={
                color list
                    \nextlist
                marks
                    \nextlist
            },
        ]
            \nextgroupplot[
                title={$20000\times20000$},
                xmode=log,
                ymode=log,
                log ticks with fixed point,
            ]

                \addplot
                    coordinates {
                        (1,100)
                        (2,83.66)
                        (4,56.04)
                        (8,81.22)
                        (16,88.07)
                        (32,81.35)
                    };
        
                \addplot
                    coordinates {
                        (1,100)
                        (2,93.08)
                        (4,52.07)
                        (8,82.80)
                        (16,92.40)
                        (32,88.49)
                    };
                
                \addplot
                    coordinates {
                        (1,100)
                        (2,84.94)
                        (4,38.84)
                        (8,81.97)
                        (16,88.51)
                        (32,79.13)
                    };
                
                \addplot
                    coordinates {
                        (1,100)
                        (2,98.18)
                        (4,68.20)
                        (8,87.12)
                        (16,96.80)
                        (32,93.02)
                    };

            \nextgroupplot[
                title={$30000\times30000$},
                xmode=log,
                ymode=log, 
                log ticks with fixed point,
            ]
                \addplot
                    coordinates {
                        (1,100)
                        (2,82.77)
                        (4,91.52)
                        (8,58.62)
                        (16,70.54)
                        (32,82.45)
                    };
                
                \addplot
                    coordinates {
                        (1,100)
                        (2,97.73)
                        (4,62.62)
                        (8,53.77)
                        (16,44.25)
                        (32,91.87)
                    };
                
                \addplot
                    coordinates {
                        (1,100)
                        (2,92.13)
                        (4,59.79)
                        (8,76.18)
                        (16,81.94)
                        (32,62.39)
                    };
                
                \addplot
                    coordinates {
                        (1,100)
                        (2,91.71)
                        (4,61.96)
                        (8,73.32)
                        (16,88.65)
                        (32,90.03)
                    };    

            \nextgroupplot[
                title={$40000\times40000$},
                xmode=log,
                ymode=log,
                log ticks with fixed point,
            ]
                \addplot
                    coordinates {
                        (1,100)
                        (2,82.26)
                        (4,45.27)
                        (8,38.64)
                        (16,59.28)
                        (32,50.49)
                    };
                
                \addplot
                    coordinates {
                        (1,100)
                        (2,91.99)
                        (4,76.37)
                        (8,71.03)
                        (16,67.58)
                        (32,56.78)
                    };
                
                \addplot
                    coordinates {
                        (1,100)
                        (2,90.10)
                        (4,49.45)
                        (8,68.96)
                        (16,82.88)
                        (32,47.99)
                    };
                
                \addplot
                    coordinates {
                        (1,100)
                        (2,97.15)
                        (4,66.47)
                        (8,68.63)
                        (16,88.07)
                        (32,87.86)
                    };
            
            \nextgroupplot[
                title={$50000\times50000$},
                xmode=log,
                ymode=log, 
                log ticks with fixed point,
            ]
                \addplot
                    coordinates {
                        (1,100)
                        (2,78.65)
                        (4,60.40)
                        (8,49.43)
                        (16,55.82)
                        (32,57.07)
                    };
                
                \addplot
                    coordinates {
                        (1,100)
                        (2,95)
                        (4,58.11)
                        (8,57.69)
                        (16,76.73)
                        (32,57.48)
                    };
                
                \addplot
                    coordinates {
                        (1,100)
                        (2,92.79)
                        (4,69.96)
                        (8,64.95)
                        (16,66.11)
                        (32,77.80)
                    };
                
                \addplot
                    coordinates {
                        (1,100)
                        (2,95.36)
                        (4,68.61)
                        (8,67.58)
                        (16,82.55)
                        (32,89.98)
                    };

            \legend{1000, 2000, 5000, 10000}
        \end{groupplot}
        \node (title) at ($(plot c1r1.center)!0.5!(plot c2r1.center)+(0,3cm)$) {with blocks};
    \end{tikzpicture}
    \ref{legend}
\end{figure}

As expected, the algorithms achieve a better performance the more CPUs they have %
access to. The same can be said about the speedup: the higher the number of %
CPUs, the faster the calculations are completed. Unfortunately, the efficiency is %
all over the place because the algorithms are not linearly scalable \cite{scalability}. %
Implementing the parallelization differently or using another algorithm altogether %
could be some of the possible fixes for this undefined yet mathematically correct %
issue.