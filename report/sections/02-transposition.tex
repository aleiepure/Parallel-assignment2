\section{Parallel matrix transposition}\label{sec:transposition}

The second task asks for the implementation of a matrix transposition algorithm in %
two different ways, written in \textit{C} and parallelized with OpenMP. The end %
goal is to compare the scalability of the two versions and evaluate their efficiency.

Transposing a matrix involves flipping it over its main diagonal, exchanging %
rows and columns, resulting in a new matrix where the rows of the original one %
become the columns of the transposed matrix and vice versa. The first algorithm %
does verbatim what described above, while the other divides the original %
matrix into blocks of fixed sizes, transposes them using the same principle, %
and finally transposes each block's content.

Both algorithms were written and tested sequentially and parallelized later %
by introducing the OpenMP directives visible in source code \ref{code:transposition}. %
The choice of which ones to use was made after reading the professor's material %
on the subject \cite{prof-slides}, OpenMP's 3.1 documentation \cite{openmp-cs, openmp-docs}, %
and trying various combinations\footnote{For an explanation of each part of the directive, %
refer to \hyperref[sec:multiplication]{Task \ref*{sec:multiplication}, Parallel %
matrix multiplication}.}.

\begin{tabularx}{\textwidth}{@{} c c c c c c c c @{}}
    \caption{\label{table:transposition}Matrix transposition - run times and bandwidth}\\
    \toprule
        \textbf{Size} & \textbf{Cores} & \textbf{Run 1 [s]} & \textbf{Run 2 [s]} & %
        \textbf{Run 3 [s]} & \textbf{Run 4 [s]} & \textbf{Average [s]} & \textbf{Bandwidth [GB/s]}\\
    \midrule
    \endhead
        \multirow{6}{*}{20000} & 1  & 6.5606 & 6.6204 & 6.6265 & 6.5600 & 6.5919 & 0.49 \\
                               & 2  & 3.4106 & 3.3133 & 3.3296 & 3.3078 & 3.3404 & 0.96 \\
                               & 4  & 2.0776 & 1.8602 & 2.0055 & 2.0716 & 2.0037 & 1.60 \\
                               & 8  & 0.9174 & 0.9229 & 0.9655 & 0.9833 & 0.9473 & 3.38 \\
                               & 16 & 0.5285 & 0.5373 & 0.5404 & 0.5687 & 0.5437 & 5.89 \\
                               & 32 & 0.3506 & 0.3423 & 0.3365 & 0.3830 & 0.3532 & 9.06 \\
    \midrule
        \multirow{6}{*}{30000} & 1  & 15.6335 & 15.5900 & 15.5903 & 16.2771 & 15.7727 & 0.46 \\
                               & 2  & 8.4734  & 8.1278  & 8.2218  & 8.2349  & 8.2645  & 0.87 \\
                               & 4  & 4.3465  & 4.8519  & 4.9496  & 4.8396  & 4.7469  & 1.52 \\
                               & 8  & 2.2628  & 2.4801  & 2.6046  & 2.2887  & 2.4091  & 2.99 \\
                               & 16 & 1.2341  & 1.8231  & 1.2558  & 1.2322  & 1.3863  & 5.19 \\
                               & 32 & 0.8240  & 0.8335  & 1.1785  & 0.7841  & 0.9050  & 7.96 \\
    \midrule
        \multirow{6}{*}{40000} & 1  & 27.9708 & 28.9593 & 27.9213 & 27.9582 & 28.2024 & 0.45 \\
                               & 2  & 14.9170 & 14.8816 & 14.8110 & 14.9570 & 14.8919 & 0.86 \\
                               & 4  & 8.4412  & 8.3528  & 8.6755  & 8.3326  & 8.4505  & 1.51 \\
                               & 8  & 4.3593  & 4.4382  & 4.2855  & 4.1860  & 4.3172  & 2.96 \\
                               & 16 & 3.2731  & 3.0261  & 2.2626  & 2.5232  & 2.7712  & 4.62 \\
                               & 32 & 3.1133  & 2.7508  & 2.5513  & 1.9700  & 2.5964  & 4.93 \\
    \midrule
        \multirow{6}{*}{50000} & 1  & 45.9321 & 44.8595 & 45.1216 & 47.1102 & 45.7558 & 0.44 \\
                               & 2  & 23.8948 & 25.4520 & 25.4178 & 25.2524 & 25.0042 & 0.80 \\
                               & 4  & 13.8669 & 13.7983 & 13.6129 & 13.5577 & 13.7090 & 1.46 \\
                               & 8  & 8.6068  & 7.5582  & 8.0404  & 8.4306  & 8.1590  & 2.45 \\
                               & 16 & 4.1942  & 5.3581  & 5.3738  & 4.5863  & 4.8781  & 4.10 \\
                               & 32 & 3.3619  & 4.7703  & 3.2806  & 3.4283  & 3.7103  & 5.39 \\
    \bottomrule
\end{tabularx}

Compilation and execution are achieved, both locally (see previous task for the %
architecture) and on the HPC cluster, via the same PBS script used for task %
\ref*{sec:multiplication}, integrated with the new commands. The outputs of the %
runs are returned in \texttt{CSV} files for easier analysis with external programs.

\subsection*{Results analysis}
The implemented code was executed with square matrices of sizes between 20000 %
and 50000, each with square blocks of 1000, 2000, 5000, and 10000. All combinations %
were run with up to 32 CPUs on \texttt{hpc-c11-node22.unitn.it}\footnote{refer to %
\hyperref[sec:multiplication]{task \ref*{sec:multiplication}} for its architecture.} %
Execution times are reported in Table \ref{table:transposition} and Table %
\ref{table:transposition-blocks} alongside the achieved bandwidth, calculated %
using Equation \ref{eq:bandwidth}. Due to the way I wrote the \textit{C} code, %
the algorithm without the blocks runs four times per matrix size per number of %
CPUs. The bandwidth reported in Table \ref{table:transposition} refers to the %
average of the runtimes.\\%

\begin{equation}
    \label{eq:bandwidth}
    \textnormal{Bandwidth}=
        \frac{%
            2 \cdot \textnormal{Rows} \cdot \textnormal{Columns} \cdot %
            \textnormal{size\_of(\textit{float})}
            }
            {\textnormal{execution time}}=\frac{2\cdot \textnormal{size}^2\cdot 4}
                                               {\textnormal{execution time}}
        \qquad \textnormal{[B/s]}
\end{equation}

\begin{tabularx}{\textwidth}{@{} c c c c c c c c @{}}
    \caption{\label{table:transposition}Matrix transposition - run times and bandwidth}\\
    \toprule
        \textbf{Size} & \textbf{Cores} & \textbf{Run 1 [s]} & \textbf{Run 2 [s]} & %
        \textbf{Run 3 [s]} & \textbf{Run 4 [s]} & \textbf{Average [s]} & \textbf{Bandwidth [GB/s]}\\
    \midrule
    \endhead
        \multirow{6}{*}{20000} & 1  & 6.5606 & 6.6204 & 6.6265 & 6.5600 & 6.5919 & 0.49 \\
                               & 2  & 3.4106 & 3.3133 & 3.3296 & 3.3078 & 3.3404 & 0.96 \\
                               & 4  & 2.0776 & 1.8602 & 2.0055 & 2.0716 & 2.0037 & 1.60 \\
                               & 8  & 0.9174 & 0.9229 & 0.9655 & 0.9833 & 0.9473 & 3.38 \\
                               & 16 & 0.5285 & 0.5373 & 0.5404 & 0.5687 & 0.5437 & 5.89 \\
                               & 32 & 0.3506 & 0.3423 & 0.3365 & 0.3830 & 0.3532 & 9.06 \\
    \midrule
        \multirow{6}{*}{30000} & 1  & 15.6335 & 15.5900 & 15.5903 & 16.2771 & 15.7727 & 0.46 \\
                               & 2  & 8.4734  & 8.1278  & 8.2218  & 8.2349  & 8.2645  & 0.87 \\
                               & 4  & 4.3465  & 4.8519  & 4.9496  & 4.8396  & 4.7469  & 1.52 \\
                               & 8  & 2.2628  & 2.4801  & 2.6046  & 2.2887  & 2.4091  & 2.99 \\
                               & 16 & 1.2341  & 1.8231  & 1.2558  & 1.2322  & 1.3863  & 5.19 \\
                               & 32 & 0.8240  & 0.8335  & 1.1785  & 0.7841  & 0.9050  & 7.96 \\
    \midrule
        \multirow{6}{*}{40000} & 1  & 27.9708 & 28.9593 & 27.9213 & 27.9582 & 28.2024 & 0.45 \\
                               & 2  & 14.9170 & 14.8816 & 14.8110 & 14.9570 & 14.8919 & 0.86 \\
                               & 4  & 8.4412  & 8.3528  & 8.6755  & 8.3326  & 8.4505  & 1.51 \\
                               & 8  & 4.3593  & 4.4382  & 4.2855  & 4.1860  & 4.3172  & 2.96 \\
                               & 16 & 3.2731  & 3.0261  & 2.2626  & 2.5232  & 2.7712  & 4.62 \\
                               & 32 & 3.1133  & 2.7508  & 2.5513  & 1.9700  & 2.5964  & 4.93 \\
    \midrule
        \multirow{6}{*}{50000} & 1  & 45.9321 & 44.8595 & 45.1216 & 47.1102 & 45.7558 & 0.44 \\
                               & 2  & 23.8948 & 25.4520 & 25.4178 & 25.2524 & 25.0042 & 0.80 \\
                               & 4  & 13.8669 & 13.7983 & 13.6129 & 13.5577 & 13.7090 & 1.46 \\
                               & 8  & 8.6068  & 7.5582  & 8.0404  & 8.4306  & 8.1590  & 2.45 \\
                               & 16 & 4.1942  & 5.3581  & 5.3738  & 4.5863  & 4.8781  & 4.10 \\
                               & 32 & 3.3619  & 4.7703  & 3.2806  & 3.4283  & 3.7103  & 5.39 \\
    \bottomrule
\end{tabularx}
\begin{tabularx}{\textwidth}{@{} c c c c c @{}}
    \caption{\label{table:transposition-blocks}Matrix transposition in blocks - run times and bandwidth}\\
\toprule
    \textbf{Size} & \textbf{Block size} & \textbf{Cores} & \textbf{Run time [s]} & \textbf{Bandwidth [GB/s]}\\
\midrule
\endhead
    \multirow{24}{*}{20000} & \multirow{6}{*}{1000} & 1  & 2.3274 & 1.37  \\
                            &                       & 2  & 1.3910 & 2.30  \\
                            &                       & 4  & 1.0382 & 3.08  \\
                            &                       & 8  & 0.3582 & 8.93  \\
                            &                       & 16 & 0.1652 & 19.37 \\
                            &                       & 32 & 0.0894 & 35.79 \\
                            \cmidrule{2-5}
                            & \multirow{6}{*}{2000} & 1  & 2.9142 & 1.10  \\
                            &                       & 2  & 1.5654 & 2.04  \\
                            &                       & 4  & 1.3991 & 2.29  \\
                            &                       & 8  & 0.4399 & 7.27  \\
                            &                       & 16 & 0.1971 & 16.23 \\
                            &                       & 32 & 0.1029 & 31.09 \\
                            \cmidrule{2-5}
                            & \multirow{6}{*}{5000} & 1  & 3.3928 & 0.94  \\
                            &                       & 2  & 1.9972 & 1.60  \\
                            &                       & 4  & 2.1841 & 1.47  \\
                            &                       & 8  & 0.5174 & 6.18  \\
                            &                       & 16 & 0.2396 & 13.36 \\
                            &                       & 32 & 0.1340 & 23.88 \\
                            \cmidrule{2-5}
                            & \multirow{6}{*}{10000} & 1  & 5.6197 & 0.57  \\
                            &                        & 2  & 2.8618 & 1.12  \\
                            &                        & 4  & 2.0601 & 1.55  \\
                            &                        & 8  & 0.8063 & 3.97  \\
                            &                        & 16 & 0.3628 & 8.82  \\
                            &                        & 32 & 0.1888 & 16.95 \\
\midrule\enlargethispage{-\baselineskip}
    \multirow{24}{*}{30000} & \multirow{6}{*}{1000} & 1  & 4.8689 & 1.48  \\
                            &                       & 2  & 2.9412 & 2.45  \\
                            &                       & 4  & 1.3300 & 5.41  \\
                            &                       & 8  & 1.0382 & 6.94  \\
                            &                       & 16 & 0.4314 & 16.69 \\
                            &                       & 32 & 0.1845 & 39.02 \\
                            \cmidrule{2-5}
                            & \multirow{6}{*}{2000} & 1  & 7.0540 & 1.02  \\
                            &                       & 2  & 3.6089 & 2.00  \\
                            &                       & 4  & 2.8164 & 2.56  \\
                            &                       & 8  & 1.6397 & 4.39  \\
                            &                       & 16 & 0.9963 & 7.23  \\
                            &                       & 32 & 0.2399 & 30.01 \\
                            \cmidrule{2-5}
                            & \multirow{6}{*}{5000} & 1  & 7.8568 & 0.92  \\
                            &                       & 2  & 4.2641 & 1.69  \\
                            &                       & 4  & 3.2850 & 2.19  \\
                            &                       & 8  & 1.2891 & 5.59  \\
                            &                       & 16 & 0.5993 & 12.01 \\
                            &                       & 32 & 0.3935 & 18.30 \\
                            \cmidrule{2-5}
                            & \multirow{6}{*}{10000} & 1  & 13.1201 & 0.55 \\
                            &                        & 2  & 7.1527  & 1.01 \\
                            &                        & 4  & 5.2946  & 1.36 \\
                            &                        & 8  & 2.2368  & 3.22 \\
                            &                        & 16 & 0.9249  & 7.78 \\
                            &                        & 32 & 0.4554  & 15.81 \\
\midrule\enlargethispage{-\baselineskip}
    \multirow{24}{*}{40000} & \multirow{6}{*}{1000} & 1  & 8.7225 & 1.47  \\
                            &                       & 2  & 5.3019 & 2.41  \\
                            &                       & 4  & 4.8171 & 2.66  \\
                            &                       & 8  & 2.8220 & 4.54  \\
                            &                       & 16 & 0.9196 & 13.92 \\
                            &                       & 32 & 0.5399 & 23.71 \\
                            \cmidrule{2-5}
                            & \multirow{6}{*}{2000} & 1  & 12.2732 & 1.04  \\
                            &                       & 2  & 6.6707  & 1.92  \\
                            &                       & 4  & 4.0177  & 3.19  \\
                            &                       & 8  & 2.1598  & 5.93  \\
                            &                       & 16 & 1.1351  & 11.28 \\
                            &                       & 32 & 0.6754  & 18.95 \\
                            \cmidrule{2-5}
                            & \multirow{6}{*}{5000} & 1  & 13.4611 & 0.95  \\
                            &                       & 2  & 7.4703  & 1.71  \\
                            &                       & 4  & 6.8054  & 1.88  \\
                            &                       & 8  & 2.4402  & 5.25  \\
                            &                       & 16 & 1.0151  & 12.61 \\
                            &                       & 32 & 0.8766  & 14.60 \\
                            \cmidrule{2-5}
                            & \multirow{6}{*}{10000} & 1  & 24.0736 & 0.53  \\
                            &                        & 2  & 12.3900 & 1.03  \\
                            &                        & 4  & 9.0544  & 1.41  \\
                            &                        & 8  & 4.3846  & 2.92  \\
                            &                        & 16 & 1.7084  & 7.49  \\
                            &                        & 32 & 0.8562  & 14.95 \\
\midrule\enlargethispage{-\baselineskip}
    \multirow{24}{*}{50000} & \multirow{6}{*}{1000} & 1  & 12.0920 & 1.65  \\
                            &                       & 2  & 7.6869  & 2.60  \\
                            &                       & 4  & 5.0051  & 4.00  \\
                            &                       & 8  & 3.0580  & 6.54  \\
                            &                       & 16 & 1.3538  & 14.77 \\
                            &                       & 32 & 0.6621  & 30.21 \\
                            \cmidrule{2-5}
                            & \multirow{6}{*}{2000} & 1  & 20.3203 & 0.98  \\
                            &                       & 2  & 10.6947 & 1.87  \\
                            &                       & 4  & 8.7424  & 2.29  \\
                            &                       & 8  & 4.4028  & 4.54  \\
                            &                       & 16 & 1.6551  & 12.08 \\
                            &                       & 32 & 1.1047  & 18.10 \\
                            \cmidrule{2-5}
                            & \multirow{6}{*}{5000} & 1  & 21.5054 & 0.93  \\
                            &                       & 2  & 11.5888 & 1.73  \\
                            &                       & 4  & 7.6854  & 2.60  \\
                            &                       & 8  & 4.1386  & 4.83  \\
                            &                       & 16 & 2.0332  & 9.84  \\
                            &                       & 32 & 0.8638  & 23.15 \\
                            \cmidrule{2-5}
                            & \multirow{6}{*}{10000} & 1  & 37.0605 & 0.54  \\
                            &                        & 2  & 19.4318 & 1.03  \\
                            &                        & 4  & 13.5048 & 1.48  \\
                            &                        & 8  & 6.8553  & 2.92  \\
                            &                        & 16 & 2.8058  & 7.13  \\
                            &                        & 32 & 1.2871  & 15.54 \\
\bottomrule
\end{tabularx}


The parallel scalability of the algorithm was evaluated using the bandwidth as %
the performance metric. Plot \ref{plot:trans-performance}, Plot \ref{plot:trans-speedup}, %
and Plot \ref{plot:trans-efficiency} visualize the performance, the speedup, and %
the efficiency respectively. 

\begin{figure}[h!tb]%
    \centering
    \captionsetup{type=plot}
    \caption{\label{plot:trans-performance}Parallel Matrix Transposition - Performance graph by size of matrices}
    \begin{tikzpicture}%
        \begin{axis}[
            title={without blocks},
            xlabel={CPUs},
            ylabel={Bandwidth [GB/s]},
            legend pos=north west,
            xtick={1,2,4,8,16,32},
            log ticks with fixed point,
            xmode=log,
            cycle multiindex* list={
                color list
                    \nextlist
                marks
                    \nextlist
            },
            width={0.5\linewidth},
        ]%
            \addplot
                coordinates {
                    (1,0.49)  
                    (2,0.96)  
                    (4,1.60)  
                    (8,3.38)  
                    (16,5.89)  
                    (32,9.06)
                };
            
            \addplot
                coordinates {
                    (1,0.46)
                    (2,0.87)
                    (4,1.52)
                    (8,2.99)
                    (16,5.19)
                    (32,7.96)
                };
            
            \addplot
                coordinates {
                    (1,0.45)
                    (2,0.86)
                    (4,1.51)
                    (8,2.96)
                    (16,4.62)
                    (32,4.93)
                };
            
            \addplot
                coordinates {
                    (1,0.44)
                    (2,0.80)
                    (4,1.46)
                    (8,2.45)
                    (16,4.10)
                    (32,5.39)
                };

            \legend{20000, 30000, 40000, 50000}
        \end{axis}
    \end{tikzpicture}
    \begin{tikzpicture}%
        \begin{groupplot}[
            group style={
                group size=2 by 2,
                group name=plot,
                xlabels at=edge bottom,
                ylabels at=edge left,
                vertical sep=1.5cm,
                horizontal sep=1.5cm,
            }, 
            width={0.45\linewidth},
            xlabel={CPUs},
            ylabel={Bandwidth [GB/s]},
            xtick={1,2,4,8,16,32},
            legend to name=legend,
            legend columns=-1,
            cycle multiindex* list={
                color list
                    \nextlist
                marks
                    \nextlist
            },
        ]
            \nextgroupplot[
                title={$20000\times20000$},
                xmode=log, 
                log ticks with fixed point,
            ]

                \addplot
                    coordinates {
                        (1,1.37)
                        (2,2.30)
                        (4,3.08)
                        (8,8.93)
                        (16,19.37)
                        (32,35.79)
                    };
        
                \addplot
                    coordinates {
                        (1,1.10)
                        (2,2.04)
                        (4,2.29)
                        (8,7.27)
                        (16,16.23)
                        (32,31.09)
                    };
                
                \addplot
                    coordinates {
                        (1,0.94)
                        (2,1.60)
                        (4,1.47)
                        (8,6.18)
                        (16,13.36)
                        (32,23.88)
                    };
                
                \addplot
                    coordinates {
                        (1,0.57)
                        (2,1.12)
                        (4,1.55)
                        (8,3.97)
                        (16,8.82)
                        (32,16.95)
                    };

            \nextgroupplot[
                title={$30000\times30000$},
                xmode=log, 
                log ticks with fixed point,
            ]
                \addplot
                    coordinates {
                        (1,1.48)
                        (2,2.45)
                        (4,5.41)
                        (8,6.94)
                        (16,16.69)
                        (32,39.02)
                    };
                
                \addplot
                    coordinates {
                        (1,1.02)
                        (2,2.00)
                        (4,2.56)
                        (8,4.39)
                        (16,7.23)
                        (32,30.01)
                    };
                
                \addplot
                    coordinates {
                        (1,0.92)
                        (2,1.69)
                        (4,2.19)
                        (8,5.59)
                        (16,12.01)
                        (32,18.30)
                    };
                
                \addplot
                    coordinates {
                        (1,0.55)
                        (2,1.01)
                        (4,1.36)
                        (8,3.22)
                        (16,7.78)
                        (32,15.81)
                    };    

            \nextgroupplot[
                title={$40000\times40000$},
                xmode=log,
                log ticks with fixed point,
            ]
                \addplot
                    coordinates {
                        (1,1.47)
                        (2,2.41)
                        (4,2.66)
                        (8,4.54)
                        (16,13.92)
                        (32,23.71)
                    };
                
                \addplot
                    coordinates {
                        (1,1.04)
                        (2,1.92)
                        (4,3.19)
                        (8,5.93)
                        (16,11.28)
                        (32,18.95)
                    };
                
                \addplot
                    coordinates {
                        (1,0.95)
                        (2,1.71)
                        (4,1.88)
                        (8,5.25)
                        (16,12.61)
                        (32,14.60)
                    };
                
                \addplot
                    coordinates {
                        (1,0.53)
                        (2,1.03)
                        (4,1.41)
                        (8,2.92)
                        (16,7.49)
                        (32,14.95)
                    };
            
            \nextgroupplot[
                title={$50000\times50000$},
                xmode=log,
                log ticks with fixed point,
            ]
                \addplot
                    coordinates {
                        (1,1.65)
                        (2,2.60)
                        (4,4.00)
                        (8,6.54)
                        (16,14.77)
                        (32,30.21)
                    };
                
                \addplot
                    coordinates {
                        (1,0.98)
                        (2,1.87)
                        (4,2.29)
                        (8,4.54)
                        (16,12.08)
                        (32,18.10)
                    };
                
                \addplot
                    coordinates {
                        (1,0.93)
                        (2,1.73)
                        (4,2.60)
                        (8,4.83)
                        (16,9.84)
                        (32,23.15)
                    };
                
                \addplot
                    coordinates {
                        (1,0.54)
                        (2,1.03)
                        (4,1.48)
                        (8,2.92)
                        (16,7.13)
                        (32,15.54)
                    };

            \legend{1000, 2000, 5000, 10000}
        \end{groupplot}
        \node (title) at ($(plot c1r1.center)!0.5!(plot c2r1.center)+(0,3cm)$) {with blocks};
    \end{tikzpicture}
    \ref{legend}
\end{figure}
\begin{figure}[h!tb]%
    \centering
    \captionsetup{type=plot}
    \caption{\label{plot:trans-speedup}Parallel Matrix Transposition - Speedup graph by size of matrices}
    \begin{tikzpicture}%
        \begin{semilogxaxis}[
            title={without blocks},
            xlabel={CPUs},
            ylabel={Speedup $[\times]$},
            legend pos=north west,
            xtick={1,2,4,8,16,32},
            log ticks with fixed point,
            cycle multiindex* list={
                color list
                    \nextlist
                marks
                    \nextlist
            },
            width={0.5\linewidth},
        ]%
            \addplot
                coordinates {
                    (1,1)
                    (2,1.97)
                    (4,3.29)
                    (8,6.96)
                    (16,12.12)
                    (32,18.67)
                };
            
            \addplot
                coordinates {
                    (1,1)
                    (2,1.91)
                    (4,3.32)
                    (8,6.55)
                    (16,11.38)
                    (32,17.43)
                };
            
            \addplot
                coordinates {
                    (1,1)
                    (2,1.89)
                    (4,3.34)
                    (8,6.53)
                    (16,10.18)
                    (32,10.86)
                };
            
            \addplot
                coordinates {
                    (1,1)
                    (2,1.83)
                    (4,3.34)
                    (8,5.61)
                    (16,9.38)
                    (32,12.33)
                };

            \legend{20000, 30000, 40000, 50000}
        \end{semilogxaxis}
    \end{tikzpicture}
    \begin{tikzpicture}%
        \begin{groupplot}[
            group style={
                group size=2 by 2,
                group name=plot,
                xlabels at=edge bottom,
                ylabels at=edge left,
                vertical sep=1.5cm,
                horizontal sep=1.5cm,
            }, 
            width={0.45\linewidth},
            xlabel={CPUs},
            ylabel={Speedup $[\times]$},
            xtick={1,2,4,8,16,32},
            legend to name=legend,
            legend columns=-1,
            cycle multiindex* list={
                color list
                    \nextlist
                marks
                    \nextlist
            },
        ]
            \nextgroupplot[
                title={$20000\times20000$},
                xmode=log, 
                log ticks with fixed point,
            ]

                \addplot
                    coordinates {
                        (1,1)
                        (2,1.67)
                        (4,2.24)
                        (8,6.50)
                        (16,14.09)
                        (32,26.03)
                    };
        
                \addplot
                    coordinates {
                        (1,1)
                        (2,1.86)
                        (4,2.08)
                        (8,6.62)
                        (16,14.78)
                        (32,28.32)
                    };
                
                \addplot
                    coordinates {
                        (1,1)
                        (2,1.70)
                        (4,1.55)
                        (8,6.56)
                        (16,14.16)
                        (32,25.32)
                    };
                
                \addplot
                    coordinates {
                        (1,1)
                        (2,1.96)
                        (4,2.73)
                        (8,6.97)
                        (16,15.49)
                        (32,29.77)
                    };

            \nextgroupplot[
                title={$30000\times30000$},
                xmode=log, 
                log ticks with fixed point,
            ]
                \addplot
                    coordinates {
                        (1,1)
                        (2,1.66)
                        (4,3.66)
                        (8,4.69)
                        (16,11.29)
                        (32,26.39)
                    };
                
                \addplot
                    coordinates {
                        (1,1)
                        (2,1.95)
                        (4,2.50)
                        (8,4.30)
                        (16,7.08)
                        (32,29.40)
                    };
                
                \addplot
                    coordinates {
                        (1,1)
                        (2,1.84)
                        (4,2.39)
                        (8,6.09)
                        (16,13.11)
                        (32,19.97)
                    };
                
                \addplot
                    coordinates {
                        (1,1)
                        (2,1.83)
                        (4,2.48)
                        (8,5.87)
                        (16,14.18)
                        (32,28.81)
                    };    

            \nextgroupplot[
                title={$40000\times40000$},
                xmode=log, 
                log ticks with fixed point,
            ]
                \addplot
                    coordinates {
                        (1,1)
                        (2,1.65)
                        (4,1.81)
                        (8,3.09)
                        (16,9.48)
                        (32,16.16)
                    };
                
                \addplot
                    coordinates {
                        (1,1)
                        (2,1.84)
                        (4,3.05)
                        (8,5.68)
                        (16,10.81)
                        (32,18.17)
                    };
                
                \addplot
                    coordinates {
                        (1,1)
                        (2,1.80)
                        (4,1.98)
                        (8,5.52)
                        (16,13.26)
                        (32,15.36)
                    };
                
                \addplot
                    coordinates {
                        (1,1)
                        (2,1.94)
                        (4,2.66)
                        (8,5.49)
                        (16,14.09)
                        (32,28.12)
                    };
            
            \nextgroupplot[
                title={$50000\times50000$},
                xmode=log, 
                log ticks with fixed point,
            ]
                \addplot
                    coordinates {
                        (1,1)
                        (2,1.57)
                        (4,2.42)
                        (8,3.95)
                        (16,8.93)
                        (32,18.26)
                    };
                
                \addplot
                    coordinates {
                        (1,1)
                        (2,1.90)
                        (4,2.32)
                        (8,4.62)
                        (16,12.28)
                        (32,18.39)
                    };
                
                \addplot
                    coordinates {
                        (1,1)
                        (2,1.86)
                        (4,2.80)
                        (8,5.20)
                        (16,10.58)
                        (32,24.90)
                    };
                
                \addplot
                    coordinates {
                        (1,1)
                        (2,1.91)
                        (4,2.74)
                        (8,5.41)
                        (16,13.21)
                        (32,28.79)
                    };

            \legend{1000, 2000, 5000, 10000}
        \end{groupplot}
        \node (title) at ($(plot c1r1.center)!0.5!(plot c2r1.center)+(0,3cm)$) {with blocks};
    \end{tikzpicture}
    \ref{legend}
\end{figure}
\begin{figure}[h!tb]%
    \centering
    \captionsetup{type=plot}
    \caption{\label{plot:trans-efficiency}Parallel Matrix Transposition - Efficiency graph by size of matrices}
    \begin{tikzpicture}%
        \begin{axis}[
            title={without blocks},
            xlabel={CPUs},
            ylabel={Efficiency [\%]},
            legend pos=south west,
            xtick={1,2,4,8,16,32},
            xmode=log,
            ymode=log,
            log ticks with fixed point,
            cycle multiindex* list={
                color list
                    \nextlist
                marks
                    \nextlist
            },
            width={0.5\linewidth},
        ]%
            \addplot
                coordinates {
                    (1,100)
                    (2,98.67)
                    (4,82.24)
                    (8,86.98)
                    (16,75.78)
                    (32,58.34)
                };
            
            \addplot
                coordinates {
                    (1,100)
                    (2,95.42)
                    (4,83.07)
                    (8,81.84)
                    (16,71.11)
                    (32,54.46)
                };
            
            \addplot
                coordinates {
                    (1,100)
                    (2,94.69)
                    (4,83.43)
                    (8,81.66)
                    (16,63.61)
                    (32,33.94)
                };
            
            \addplot
                coordinates {
                    (1,100)
                    (2,91.50)
                    (4,83.44)
                    (8,70.10)
                    (16,58.62)
                    (32,38.54)
                };

            \legend{20000, 30000, 40000, 50000}
        \end{axis}
    \end{tikzpicture}
    \begin{tikzpicture}%
        \begin{groupplot}[
            group style={
                group size=2 by 2,
                group name=plot,
                xlabels at=edge bottom,
                ylabels at=edge left,
                vertical sep=1.5cm,
                horizontal sep=1.5cm,
            }, 
            width={0.45\linewidth},
            xlabel={CPUs},
            ylabel={Efficiency [\%]},
            xtick={1,2,4,8,16,32},
            legend to name=legend,
            legend columns=-1,
            cycle multiindex* list={
                color list
                    \nextlist
                marks
                    \nextlist
            },
        ]
            \nextgroupplot[
                title={$20000\times20000$},
                xmode=log,
                ymode=log,
                log ticks with fixed point,
            ]

                \addplot
                    coordinates {
                        (1,100)
                        (2,83.66)
                        (4,56.04)
                        (8,81.22)
                        (16,88.07)
                        (32,81.35)
                    };
        
                \addplot
                    coordinates {
                        (1,100)
                        (2,93.08)
                        (4,52.07)
                        (8,82.80)
                        (16,92.40)
                        (32,88.49)
                    };
                
                \addplot
                    coordinates {
                        (1,100)
                        (2,84.94)
                        (4,38.84)
                        (8,81.97)
                        (16,88.51)
                        (32,79.13)
                    };
                
                \addplot
                    coordinates {
                        (1,100)
                        (2,98.18)
                        (4,68.20)
                        (8,87.12)
                        (16,96.80)
                        (32,93.02)
                    };

            \nextgroupplot[
                title={$30000\times30000$},
                xmode=log,
                ymode=log, 
                log ticks with fixed point,
            ]
                \addplot
                    coordinates {
                        (1,100)
                        (2,82.77)
                        (4,91.52)
                        (8,58.62)
                        (16,70.54)
                        (32,82.45)
                    };
                
                \addplot
                    coordinates {
                        (1,100)
                        (2,97.73)
                        (4,62.62)
                        (8,53.77)
                        (16,44.25)
                        (32,91.87)
                    };
                
                \addplot
                    coordinates {
                        (1,100)
                        (2,92.13)
                        (4,59.79)
                        (8,76.18)
                        (16,81.94)
                        (32,62.39)
                    };
                
                \addplot
                    coordinates {
                        (1,100)
                        (2,91.71)
                        (4,61.96)
                        (8,73.32)
                        (16,88.65)
                        (32,90.03)
                    };    

            \nextgroupplot[
                title={$40000\times40000$},
                xmode=log,
                ymode=log,
                log ticks with fixed point,
            ]
                \addplot
                    coordinates {
                        (1,100)
                        (2,82.26)
                        (4,45.27)
                        (8,38.64)
                        (16,59.28)
                        (32,50.49)
                    };
                
                \addplot
                    coordinates {
                        (1,100)
                        (2,91.99)
                        (4,76.37)
                        (8,71.03)
                        (16,67.58)
                        (32,56.78)
                    };
                
                \addplot
                    coordinates {
                        (1,100)
                        (2,90.10)
                        (4,49.45)
                        (8,68.96)
                        (16,82.88)
                        (32,47.99)
                    };
                
                \addplot
                    coordinates {
                        (1,100)
                        (2,97.15)
                        (4,66.47)
                        (8,68.63)
                        (16,88.07)
                        (32,87.86)
                    };
            
            \nextgroupplot[
                title={$50000\times50000$},
                xmode=log,
                ymode=log, 
                log ticks with fixed point,
            ]
                \addplot
                    coordinates {
                        (1,100)
                        (2,78.65)
                        (4,60.40)
                        (8,49.43)
                        (16,55.82)
                        (32,57.07)
                    };
                
                \addplot
                    coordinates {
                        (1,100)
                        (2,95)
                        (4,58.11)
                        (8,57.69)
                        (16,76.73)
                        (32,57.48)
                    };
                
                \addplot
                    coordinates {
                        (1,100)
                        (2,92.79)
                        (4,69.96)
                        (8,64.95)
                        (16,66.11)
                        (32,77.80)
                    };
                
                \addplot
                    coordinates {
                        (1,100)
                        (2,95.36)
                        (4,68.61)
                        (8,67.58)
                        (16,82.55)
                        (32,89.98)
                    };

            \legend{1000, 2000, 5000, 10000}
        \end{groupplot}
        \node (title) at ($(plot c1r1.center)!0.5!(plot c2r1.center)+(0,3cm)$) {with blocks};
    \end{tikzpicture}
    \ref{legend}
\end{figure}

As expected, the algorithms achieve a better performance the more CPUs they have %
access to. The same can be said about the speedup: the higher the number of %
CPUs, the faster the calculations are completed. Unfortunately, the efficiency is %
all over the place because the algorithms are not linearly scalable \cite{scalability}. %
Implementing the parallelization differently or using another algorithm altogether %
could be some of the possible fixes for this undefined yet mathematically correct %
issue.